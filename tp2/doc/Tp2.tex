% Facultad de Ingenier\'ia, Universidad de Buenos Aires
% 75.59 Técnicas de Programación Concurrente I

\documentclass[a4paper,12pt,titlepage]{article}
\usepackage[paperwidth=180mm,paperheight=285mm,left=1.5cm,top=4cm,right=1.5cm,bottom=2cm,head=2.0cm,includefoot]{geometry}
\usepackage[spanish]{babel}
%\usepackage[latin1]{inputenc}
\usepackage[utf8]{inputenc}
\usepackage{lscape}
\usepackage{graphicx}
\usepackage{fancyhdr}
\usepackage{rotating}
%\graphicspath{{../}}

\usepackage{listingsutf8}

\title{75.59 Técnicas de Programación Concurrente I, Trabajo Práctico 2}
\author{Torres, Miguel \and Montoya, Diego \and Garay, Ignacio}

\lhead{\includegraphics[scale=0.06]{./logo_fiuba.pdf}}
\chead{ 75.59 Técnicas de Programación Concurrente I }
\rhead{}

\lfoot{Garay - Montoya - Torres}
\rfoot{\thepage}
\cfoot{$2^{do}$ Cuatrimestre 2013}

\begin{document}

\thispagestyle{empty}
% T\'itulo del documento.
\begin{center}
\includegraphics{./logo-fiuba.png}\\
\vspace{1cm}
\textsc{\LARGE Universidad de Buenos Aires}\\[0.3cm]
\textsc{\LARGE Facultad de Ingenier\'ia}\\[1.2cm]
\textsc{\Large 75.59 - Técnicas de Programación Concurrente I}\\[0.3cm]
\end{center}

\begin{flushright}
{\large
Montoya, Diego -- 91939\\
Torres, Miguel -- 91396\\
Garay, Ignacio -- 92265\\
\vspace{2cm}
$2^{do}$ cuatrimestre de 2013}
\end{flushright}

\pagestyle{fancy}
\setcounter{page}{1}
\newpage

\tableofcontents
\newpage

\footnotesize
\section{Análisis}
En el análisis del trabajo se identificaron las siguientes identidades del dominio:\\
\begin{itemize}
\item Cliente
\item Servidor
\item Receptor de Clientes
\item Resolvedor de Paquetes\\
\end{itemize} 

El Servidor es la clase que inicia todos los recursos necesarios para una tener una conversación, creando además
al Receptor de Clientes y al Resolvedor de Paquetes. El Servidor además es el encargado de de gestionar las 
conversaciones, permitiendo agregar nuevas y administrando las ya existentes.

Cuando un Cliente es iniciado, es el Receptor de Clientes el encargado de iniciar la sesión el mismo con el Servidor, 
dejando todo listo para que el Cliente pueda crear una nueva conversación o unirse a una existente.

Una vez que algunos clientes forman parte de la misma conversación, los mensajes entre ellos enviados son capturados
por el Resolvedor de Paquetes, el cual se encarga de distribuir dichos mensajes a las conversaciones correspondientes.

Puede verse entonces que el Cliente actúa como productor de mensajes, los cuales son consumidos por el Resolvedor
de paquetes, de manera tal que cada uno de ellos llegue a los otros clientes que forman parte de la conversación.

\subsection{Casos de Uso}
Se identificaron además los siguientes casos de uso, desde el punto de vista del usuario:\\
\begin{figure}[h!]
\centering
\includegraphics[width=0.6\textwidth]{CasosDeUso.png}
\caption{Diagrama de casos de uso del Cliente}
\label{fig:casos_uso}
\end{figure}

\begin{itemize}
\item \textbf{Crear conversación}\\ 
  Es el caso de uso en el cual un Cliente conectado al Servidor decide iniciar una nueva conversación, en la cual inicialmente
  será el único miembro, hasta que otro Cliente decida unirse para conversar
\item \textbf{Consultar conversaciones}\\
  Es el caso de uso en el cual el Cliente conectado al Servidor decide realizar una consulta sobre las conversaciones
  existentes actualmente en el Servidor
\item \textbf{Unirse a conversación}\\
  Es el caso de uso en el cual un Cliente luego de consultar las conversaciones existentes en el Servidor decide además
  unirse a una de ellas\\
\end{itemize} 


\newpage
\section{Diseño}

Para la resolución de la concurrencia en la aplicacion se implementaron varias herramientas como Fifo, Pipe, Lock , 
Memoria Compartida, Señales y Sockets.

Se crearon los siguientes procesos, uno encargado de la generacion de los recursos y administración de las conversaciones, llamado 
appServidor, este último es el encargado de crear el proceso llamados appResolvedor, encargado de gestionar y resolver la llegada de 
los mensajes a su destino\\

Por último existe el proceso llamado appCliente, el cual es creado cuando un usuario inicia un Cliente y le permite interactuar con
el Servidor para unirse a una conversación o iniciar una nueva\\


Se implemento el mismo sistema de log que en primer proyecto, el cual divide los sucesos en diferentes categorías, las cuales son:
\begin{itemize}
\item Info: informacion corriente de los pasos de la ejecucion.
\item Debug: informacion de debug.
\item Fatal: indica una excepcion lanzada.
\item Warning: informacion de advertencia sobre algun comportamiento anormal.
\item Error: indica un error critico en la aplicacion
\end{itemize}
El archivo de salida de Log esta resguardado por un Lock para su correcta escritura por los distintos procesos que vuelcan su información.

\newpage
\section{Diagramas}
\subsection{Diagramas de estados}


\newpage
\subsection{Diagrama de clases}
\begin{figure}[h!]
\centering
\includegraphics[width=0.8\textwidth]{clases.png}
\caption{Diagrama de clases de dominio más los mecanismos IPCs más representativos}
\label{fig:clases}
\end{figure}

\newpage
\section{Integración}


\end{document}

